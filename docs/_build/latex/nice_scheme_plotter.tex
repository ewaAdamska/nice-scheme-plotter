%% Generated by Sphinx.
\def\sphinxdocclass{report}
\documentclass[letterpaper,10pt,english]{sphinxmanual}
\ifdefined\pdfpxdimen
   \let\sphinxpxdimen\pdfpxdimen\else\newdimen\sphinxpxdimen
\fi \sphinxpxdimen=.75bp\relax

\PassOptionsToPackage{warn}{textcomp}
\usepackage[utf8]{inputenc}
\ifdefined\DeclareUnicodeCharacter
% support both utf8 and utf8x syntaxes
\edef\sphinxdqmaybe{\ifdefined\DeclareUnicodeCharacterAsOptional\string"\fi}
  \DeclareUnicodeCharacter{\sphinxdqmaybe00A0}{\nobreakspace}
  \DeclareUnicodeCharacter{\sphinxdqmaybe2500}{\sphinxunichar{2500}}
  \DeclareUnicodeCharacter{\sphinxdqmaybe2502}{\sphinxunichar{2502}}
  \DeclareUnicodeCharacter{\sphinxdqmaybe2514}{\sphinxunichar{2514}}
  \DeclareUnicodeCharacter{\sphinxdqmaybe251C}{\sphinxunichar{251C}}
  \DeclareUnicodeCharacter{\sphinxdqmaybe2572}{\textbackslash}
\fi
\usepackage{cmap}
\usepackage[T1]{fontenc}
\usepackage{amsmath,amssymb,amstext}
\usepackage{babel}
\usepackage{times}
\usepackage[Bjarne]{fncychap}
\usepackage{sphinx}

\fvset{fontsize=\small}
\usepackage{geometry}

% Include hyperref last.
\usepackage{hyperref}
% Fix anchor placement for figures with captions.
\usepackage{hypcap}% it must be loaded after hyperref.
% Set up styles of URL: it should be placed after hyperref.
\urlstyle{same}
\addto\captionsenglish{\renewcommand{\contentsname}{Contents:}}

\addto\captionsenglish{\renewcommand{\figurename}{Fig.}}
\addto\captionsenglish{\renewcommand{\tablename}{Table}}
\addto\captionsenglish{\renewcommand{\literalblockname}{Listing}}

\addto\captionsenglish{\renewcommand{\literalblockcontinuedname}{continued from previous page}}
\addto\captionsenglish{\renewcommand{\literalblockcontinuesname}{continues on next page}}
\addto\captionsenglish{\renewcommand{\sphinxnonalphabeticalgroupname}{Non-alphabetical}}
\addto\captionsenglish{\renewcommand{\sphinxsymbolsname}{Symbols}}
\addto\captionsenglish{\renewcommand{\sphinxnumbersname}{Numbers}}

\addto\extrasenglish{\def\pageautorefname{page}}

\setcounter{tocdepth}{1}



\title{nice\_scheme\_plotter Documentation}
\date{Nov 30, 2018}
\release{01}
\author{Ewa Adamska}
\newcommand{\sphinxlogo}{\vbox{}}
\renewcommand{\releasename}{Release}
\makeindex
\begin{document}

\pagestyle{empty}
\maketitle
\pagestyle{plain}
\sphinxtableofcontents
\pagestyle{normal}
\phantomsection\label{\detokenize{nice_scheme_plotter::doc}}

\phantomsection\label{\detokenize{nice_scheme_plotter:module-database_reader}}\index{database\_reader (module)@\spxentry{database\_reader}\spxextra{module}}\index{Level (class in database\_reader)@\spxentry{Level}\spxextra{class in database\_reader}}

\begin{fulllineitems}
\phantomsection\label{\detokenize{nice_scheme_plotter:database_reader.Level}}\pysiglinewithargsret{\sphinxbfcode{\sphinxupquote{class }}\sphinxcode{\sphinxupquote{database\_reader.}}\sphinxbfcode{\sphinxupquote{Level}}}{\emph{energy}, \emph{spinValue=None}, \emph{parity=None}, \emph{lifetime=None}}{}
This is class for excited nuclear levels, which contains all information
about level itself and its plotting style.
\begin{quote}\begin{description}
\item[{Parameters}] \leavevmode\begin{description}
\item[{\sphinxstylestrong{energy}}] \leavevmode{[}float{]}
Excited level energy.

\item[{\sphinxstylestrong{spinValue}}] \leavevmode{[}str{]}
Spin value as a string ‘1/2’, ‘5/2’, etc.

\item[{\sphinxstylestrong{parity}}] \leavevmode{[}str \{‘+’, ‘-‘, ‘’\} or None{]}
Level parity.

\item[{\sphinxstylestrong{lifetime}}] \leavevmode{[}float{]}
Excited level lifetime.

\end{description}

\item[{Attributes}] \leavevmode\begin{description}
\item[{\sphinxstylestrong{energy}}] \leavevmode{[}float{]}
Excited level energy

\item[{\sphinxstylestrong{spinValue}}] \leavevmode{[}str{]}
Excited level spin, represented by a string. Example: ‘1/2’, ‘5/2’, etc.

\item[{\sphinxstylestrong{parity}}] \leavevmode{[}str \{‘-‘, ‘+’, ‘’\}{]}
Excited level parity.

\item[{\sphinxstylestrong{level\_linewidth: float}}] \leavevmode
Level linewidth on the plot, default value is 0.5

\item[{\sphinxstylestrong{color}}] \leavevmode{[}str \{‘black’, ‘red’, ‘green’, etc.\} or RGB code{]}
Level line color. Default value is ‘black’.

\item[{\sphinxstylestrong{linestyle}}] \leavevmode{[}str \{‘solid’, ‘dashed’\}{]}
Level linestyle.

\item[{\sphinxstylestrong{lifetime}}] \leavevmode{[}float{]}
Level lifetime.

\end{description}

\end{description}\end{quote}
\subsubsection*{Methods}


\begin{savenotes}\sphinxattablestart
\centering
\begin{tabulary}{\linewidth}[t]{|T|T|}
\hline

\sphinxstylestrong{highlight(linewidth=4, color=’red’)}
&
Changes instance’s linewidth and color attributes.
\\
\hline
\end{tabulary}
\par
\sphinxattableend\end{savenotes}

\end{fulllineitems}

\index{Transition (class in database\_reader)@\spxentry{Transition}\spxextra{class in database\_reader}}

\begin{fulllineitems}
\phantomsection\label{\detokenize{nice_scheme_plotter:database_reader.Transition}}\pysiglinewithargsret{\sphinxbfcode{\sphinxupquote{class }}\sphinxcode{\sphinxupquote{database\_reader.}}\sphinxbfcode{\sphinxupquote{Transition}}}{\emph{gammaEnergy}, \emph{from\_lvl}, \emph{to\_lvl}, \emph{gammaEnergy\_err=None}, \emph{intensity=None}, \emph{instensity\_err=None}}{}~\begin{description}
\item[{This is class for transitions of the nuclear states with emission of a gamma ray. The class instance contains all}] \leavevmode
information about transition itself and its plotting style.

\end{description}
\begin{quote}\begin{description}
\item[{Parameters}] \leavevmode\begin{description}
\item[{\sphinxstylestrong{gammaEnergy}}] \leavevmode{[}float{]}
Excited level energy.

\item[{\sphinxstylestrong{from\_lvl}}] \leavevmode{[}float{]}
Energy of the state in which the nuclei was \sphinxstylestrong{before} gamma transition.

\item[{\sphinxstylestrong{to\_lvl}}] \leavevmode{[}float{]}
Energy of the state in which the nuclei was \sphinxstylestrong{after} gamma transition.

\item[{\sphinxstylestrong{gammaEnergy\_err}}] \leavevmode{[}float{]}
Excited level energy error value (default value is None).

\item[{\sphinxstylestrong{intensity}}] \leavevmode{[}float{]}
Intensity of the transition (default value is None).

\item[{\sphinxstylestrong{intensity\_err}}] \leavevmode{[}float{]}
Energy of the level in which the nuclei was before gamma transition (default value is None).

\end{description}

\item[{Attributes}] \leavevmode\begin{description}
\item[{\sphinxstylestrong{gammaEnergy}}] \leavevmode{[}float{]}
\item[{\sphinxstylestrong{from\_lvl}}] \leavevmode{[}float{]}
\item[{\sphinxstylestrong{to\_lvl}}] \leavevmode{[}float{]}
\item[{\sphinxstylestrong{gammaEnergy\_err}}] \leavevmode{[}float{]}
\item[{\sphinxstylestrong{intensity}}] \leavevmode{[}float{]}
\item[{\sphinxstylestrong{instensity\_err}}] \leavevmode{[}float{]}
\item[{\sphinxstylestrong{transition\_linewidth: float}}] \leavevmode
Transition linewidth on the plot, default value is 0.001. Be careful, there is different scale of width in use, in comparison to class Level.

\item[{\sphinxstylestrong{color}}] \leavevmode{[}str \{‘black’, ‘red’, ‘green’, etc.\} or RGB code{]}
Level line color. Default value is ‘black’.

\item[{\sphinxstylestrong{linestyle}}] \leavevmode{[}str \{‘solid’, ‘dashed’\}{]}
Level linestyle.

\item[{\sphinxstylestrong{lifetime}}] \leavevmode{[}float{]}
\end{description}

\end{description}\end{quote}
\subsubsection*{Methods}


\begin{savenotes}\sphinxatlongtablestart\begin{longtable}{\X{1}{2}\X{1}{2}}
\hline

\endfirsthead

\multicolumn{2}{c}%
{\makebox[0pt]{\sphinxtablecontinued{\tablename\ \thetable{} -- continued from previous page}}}\\
\hline

\endhead

\hline
\multicolumn{2}{r}{\makebox[0pt][r]{\sphinxtablecontinued{Continued on next page}}}\\
\endfoot

\endlastfoot

{\hyperref[\detokenize{nice_scheme_plotter:database_reader.Transition.transitionDescription}]{\sphinxcrossref{\sphinxcode{\sphinxupquote{transitionDescription}}}}}()
&
Returns transition description as a string.
\\
\hline
\end{longtable}\sphinxatlongtableend\end{savenotes}
\index{transitionDescription() (database\_reader.Transition method)@\spxentry{transitionDescription()}\spxextra{database\_reader.Transition method}}

\begin{fulllineitems}
\phantomsection\label{\detokenize{nice_scheme_plotter:database_reader.Transition.transitionDescription}}\pysiglinewithargsret{\sphinxbfcode{\sphinxupquote{transitionDescription}}}{}{}
Returns transition description as a string.
\begin{quote}\begin{description}
\item[{Returns}] \leavevmode
str ‘E (dE)  I (dI)’

\end{description}\end{quote}

\end{fulllineitems}


\end{fulllineitems}

\index{Database\_csv (class in database\_reader)@\spxentry{Database\_csv}\spxextra{class in database\_reader}}

\begin{fulllineitems}
\phantomsection\label{\detokenize{nice_scheme_plotter:database_reader.Database_csv}}\pysiglinewithargsret{\sphinxbfcode{\sphinxupquote{class }}\sphinxcode{\sphinxupquote{database\_reader.}}\sphinxbfcode{\sphinxupquote{Database\_csv}}}{\emph{lvlFileName}, \emph{transitionsFileName}}{}
Create database from csv file.
\begin{quote}\begin{description}
\item[{Parameters}] \leavevmode\begin{description}
\item[{\sphinxstylestrong{lvlFileName}}] \leavevmode{[}str{]}
File which contains lvls description.

\item[{\sphinxstylestrong{transitionsFileName}}] \leavevmode{[}str{]}
File which contains transitions description.

\end{description}

\item[{Attributes}] \leavevmode\begin{description}
\item[{\sphinxstylestrong{levels}}] \leavevmode{[}pandas.DataFrame{]}
Contains levels information.

\item[{\sphinxstylestrong{transitions}}] \leavevmode{[}pandas.DataFrame{]}
Contains transitions information.

\end{description}

\end{description}\end{quote}
\subsubsection*{Methods}


\begin{savenotes}\sphinxatlongtablestart\begin{longtable}{\X{1}{2}\X{1}{2}}
\hline

\endfirsthead

\multicolumn{2}{c}%
{\makebox[0pt]{\sphinxtablecontinued{\tablename\ \thetable{} -- continued from previous page}}}\\
\hline

\endhead

\hline
\multicolumn{2}{r}{\makebox[0pt][r]{\sphinxtablecontinued{Continued on next page}}}\\
\endfoot

\endlastfoot

{\hyperref[\detokenize{nice_scheme_plotter:database_reader.Database_csv.levelsPackage}]{\sphinxcrossref{\sphinxcode{\sphinxupquote{levelsPackage}}}}}()
&
Creates dictionary of Level\_objects
\\
\hline
{\hyperref[\detokenize{nice_scheme_plotter:database_reader.Database_csv.transitionsPackage}]{\sphinxcrossref{\sphinxcode{\sphinxupquote{transitionsPackage}}}}}()
&
Creates dictionary of Transition\_objects
\\
\hline
\end{longtable}\sphinxatlongtableend\end{savenotes}


\begin{savenotes}\sphinxattablestart
\centering
\begin{tabulary}{\linewidth}[t]{|T|T|}
\hline

\sphinxstylestrong{slice}
&\\
\hline
\end{tabulary}
\par
\sphinxattableend\end{savenotes}
\index{levelsPackage() (database\_reader.Database\_csv method)@\spxentry{levelsPackage()}\spxextra{database\_reader.Database\_csv method}}

\begin{fulllineitems}
\phantomsection\label{\detokenize{nice_scheme_plotter:database_reader.Database_csv.levelsPackage}}\pysiglinewithargsret{\sphinxbfcode{\sphinxupquote{levelsPackage}}}{}{}
Creates dictionary of Level\_objects
\begin{quote}\begin{description}
\item[{Returns}] \leavevmode
dictionary of Level\_objects with keys equal to energy \{‘energy’ : Level\_object \}

\end{description}\end{quote}

\end{fulllineitems}

\index{transitionsPackage() (database\_reader.Database\_csv method)@\spxentry{transitionsPackage()}\spxextra{database\_reader.Database\_csv method}}

\begin{fulllineitems}
\phantomsection\label{\detokenize{nice_scheme_plotter:database_reader.Database_csv.transitionsPackage}}\pysiglinewithargsret{\sphinxbfcode{\sphinxupquote{transitionsPackage}}}{}{}
Creates dictionary of Transition\_objects
\begin{quote}\begin{description}
\item[{Returns}] \leavevmode
dictionary of Transition objects with keys equal to the transition’s energy \{‘energy’ : Transition\_object \}

\end{description}\end{quote}

\end{fulllineitems}


\end{fulllineitems}

\index{Database\_xlsx (class in database\_reader)@\spxentry{Database\_xlsx}\spxextra{class in database\_reader}}

\begin{fulllineitems}
\phantomsection\label{\detokenize{nice_scheme_plotter:database_reader.Database_xlsx}}\pysiglinewithargsret{\sphinxbfcode{\sphinxupquote{class }}\sphinxcode{\sphinxupquote{database\_reader.}}\sphinxbfcode{\sphinxupquote{Database\_xlsx}}}{\emph{databaseFileName}}{}
Create database from xlsx file. This classs inherited methods from Database\_csv class.
\begin{quote}\begin{description}
\item[{Parameters}] \leavevmode\begin{description}
\item[{\sphinxstylestrong{databaseFileName}}] \leavevmode{[}str{]}
File which contains lvls description.

\end{description}

\item[{Attributes}] \leavevmode\begin{description}
\item[{\sphinxstylestrong{levels}}] \leavevmode{[}pandas.DataFrame{]}
Contains levels information.

\item[{\sphinxstylestrong{transitions}}] \leavevmode{[}pandas.DataFrame{]}
Contains transitions information.

\end{description}

\end{description}\end{quote}
\subsubsection*{Methods}


\begin{savenotes}\sphinxatlongtablestart\begin{longtable}{\X{1}{2}\X{1}{2}}
\hline

\endfirsthead

\multicolumn{2}{c}%
{\makebox[0pt]{\sphinxtablecontinued{\tablename\ \thetable{} -- continued from previous page}}}\\
\hline

\endhead

\hline
\multicolumn{2}{r}{\makebox[0pt][r]{\sphinxtablecontinued{Continued on next page}}}\\
\endfoot

\endlastfoot

\sphinxcode{\sphinxupquote{levelsPackage}}()
&
Creates dictionary of Level\_objects
\\
\hline
\sphinxcode{\sphinxupquote{transitionsPackage}}()
&
Creates dictionary of Transition\_objects
\\
\hline
\end{longtable}\sphinxatlongtableend\end{savenotes}


\begin{savenotes}\sphinxattablestart
\centering
\begin{tabulary}{\linewidth}[t]{|T|T|}
\hline

\sphinxstylestrong{slice}
&\\
\hline
\end{tabulary}
\par
\sphinxattableend\end{savenotes}

\end{fulllineitems}

\phantomsection\label{\detokenize{nice_scheme_plotter:module-nice_scheme_plotter}}\index{nice\_scheme\_plotter (module)@\spxentry{nice\_scheme\_plotter}\spxextra{module}}\index{Scheme (class in nice\_scheme\_plotter)@\spxentry{Scheme}\spxextra{class in nice\_scheme\_plotter}}

\begin{fulllineitems}
\phantomsection\label{\detokenize{nice_scheme_plotter:nice_scheme_plotter.Scheme}}\pysiglinewithargsret{\sphinxbfcode{\sphinxupquote{class }}\sphinxcode{\sphinxupquote{nice\_scheme\_plotter.}}\sphinxbfcode{\sphinxupquote{Scheme}}}{\emph{**kwargs}}{}
Creates scheme object with various methods for plotting gamma transitions scheme of the excited nuclei.

…
\begin{quote}\begin{description}
\item[{Attributes}] \leavevmode\begin{description}
\item[{\sphinxstylestrong{figureWidth}}] \leavevmode{[}float{]}
Class attribute. Output scheme window/canvas width.

\item[{\sphinxstylestrong{figureLength}}] \leavevmode{[}float{]}
Class attribute. Output scheme window/canvas length.

\item[{\sphinxstylestrong{dpi}}] \leavevmode{[}int{]}
Class attribute. Output scheme window/canvas dpi factor.

\item[{\sphinxstylestrong{fontSize}}] \leavevmode{[}int{]}
Class attribute. Level labels font size.

\item[{\sphinxstylestrong{transition\_fontSize}}] \leavevmode{[}int{]}
Class attribute. Transitions labels font size.

\item[{\sphinxstylestrong{spinAnnotationWidthFactor}}] \leavevmode{[}float{]}
Class attribute. Part of scheme plot width which will be taken by left sided annotation (spin and parity part).

\item[{\sphinxstylestrong{energyAnnotationWidthFactor}}] \leavevmode{[}float{]}
Class attribute. Part of scheme plot width which will be taken by right sided annotation (level energy).

\item[{\sphinxstylestrong{spinAnnotationSlopeFactor}}] \leavevmode{[}float{]}
Class attribute. Part of scheme plot width which will be taken for slope \sphinxstylestrong{on the left side}, when annotation and level line splitting
is needed (this is needed when bunch of levels is closer to each other than annotation height.

\item[{\sphinxstylestrong{energyAnnotationSlopeFactor}}] \leavevmode{[}float{]}
Class attribute. Part of scheme plot width which will be taken for slope \sphinxstylestrong{on the right side}, when annotation and level line splitting
is needed (this is needed when bunch of levels is closer to each other than annotation height.

\item[{\sphinxstylestrong{transtitionsSpacingFactor}}] \leavevmode{[}float{]}
Class attribute. Part of scheme plot width which will be taken as gap between transition arrows.

\end{description}

\end{description}\end{quote}
\subsubsection*{Methods}


\begin{savenotes}\sphinxatlongtablestart\begin{longtable}{\X{1}{2}\X{1}{2}}
\hline

\endfirsthead

\multicolumn{2}{c}%
{\makebox[0pt]{\sphinxtablecontinued{\tablename\ \thetable{} -- continued from previous page}}}\\
\hline

\endhead

\hline
\multicolumn{2}{r}{\makebox[0pt][r]{\sphinxtablecontinued{Continued on next page}}}\\
\endfoot

\endlastfoot

{\hyperref[\detokenize{nice_scheme_plotter:nice_scheme_plotter.Scheme.addLevel}]{\sphinxcrossref{\sphinxcode{\sphinxupquote{addLevel}}}}}(Level\_object)
&
Plots level on the scheme.
\\
\hline
{\hyperref[\detokenize{nice_scheme_plotter:nice_scheme_plotter.Scheme.addLevelsPackage}]{\sphinxcrossref{\sphinxcode{\sphinxupquote{addLevelsPackage}}}}}(levelsPackage)
&
Plots all levels from the levels package (see more about levelsPackage).
\\
\hline
{\hyperref[\detokenize{nice_scheme_plotter:nice_scheme_plotter.Scheme.addNucleiName}]{\sphinxcrossref{\sphinxcode{\sphinxupquote{addNucleiName}}}}}({[}nucleiName{]})
&
Adds nuclei name to the decay scheme.
\\
\hline
{\hyperref[\detokenize{nice_scheme_plotter:nice_scheme_plotter.Scheme.addTransition}]{\sphinxcrossref{\sphinxcode{\sphinxupquote{addTransition}}}}}(Transition\_object)
&
Plots transition on the scheme.
\\
\hline
{\hyperref[\detokenize{nice_scheme_plotter:nice_scheme_plotter.Scheme.addTransitionsPackage}]{\sphinxcrossref{\sphinxcode{\sphinxupquote{addTransitionsPackage}}}}}(transitionsPackage)
&
Plots all transition from the transitions package (see more about transitionsPackage).
\\
\hline
{\hyperref[\detokenize{nice_scheme_plotter:nice_scheme_plotter.Scheme.enableLatex}]{\sphinxcrossref{\sphinxcode{\sphinxupquote{enableLatex}}}}}()
&
Enables LaTeX rendering for all strings in the scheme plot.
\\
\hline
{\hyperref[\detokenize{nice_scheme_plotter:nice_scheme_plotter.Scheme.save}]{\sphinxcrossref{\sphinxcode{\sphinxupquote{save}}}}}({[}fileName{]})
&
Saves plot to the file.
\\
\hline
{\hyperref[\detokenize{nice_scheme_plotter:nice_scheme_plotter.Scheme.show}]{\sphinxcrossref{\sphinxcode{\sphinxupquote{show}}}}}()
&
Shows resulting scheme.
\\
\hline
\end{longtable}\sphinxatlongtableend\end{savenotes}
\index{addLevel() (nice\_scheme\_plotter.Scheme method)@\spxentry{addLevel()}\spxextra{nice\_scheme\_plotter.Scheme method}}

\begin{fulllineitems}
\phantomsection\label{\detokenize{nice_scheme_plotter:nice_scheme_plotter.Scheme.addLevel}}\pysiglinewithargsret{\sphinxbfcode{\sphinxupquote{addLevel}}}{\emph{Level\_object}}{}
Plots level on the scheme.
\begin{quote}\begin{description}
\item[{Param}] \leavevmode
Level\_object

\end{description}\end{quote}

\end{fulllineitems}

\index{addLevelsPackage() (nice\_scheme\_plotter.Scheme method)@\spxentry{addLevelsPackage()}\spxextra{nice\_scheme\_plotter.Scheme method}}

\begin{fulllineitems}
\phantomsection\label{\detokenize{nice_scheme_plotter:nice_scheme_plotter.Scheme.addLevelsPackage}}\pysiglinewithargsret{\sphinxbfcode{\sphinxupquote{addLevelsPackage}}}{\emph{levelsPackage}}{}
Plots all levels from the levels package (see more about levelsPackage).
\begin{quote}\begin{description}
\item[{Param}] \leavevmode
levelsPackage

\end{description}\end{quote}

\end{fulllineitems}

\index{addNucleiName() (nice\_scheme\_plotter.Scheme method)@\spxentry{addNucleiName()}\spxextra{nice\_scheme\_plotter.Scheme method}}

\begin{fulllineitems}
\phantomsection\label{\detokenize{nice_scheme_plotter:nice_scheme_plotter.Scheme.addNucleiName}}\pysiglinewithargsret{\sphinxbfcode{\sphinxupquote{addNucleiName}}}{\emph{nucleiName='\$\textasciicircum{}\{63\}\$Ni'}}{}
Adds nuclei name to the decay scheme.
\begin{quote}\begin{description}
\item[{Param}] \leavevmode
nucleiName : \sphinxstyleemphasis{str} (best option is to use LaTeX typing method. Example: nucleiName=r’\$\textasciicircum{}\{63\}\$Ni’)

\end{description}\end{quote}

\end{fulllineitems}

\index{addTransition() (nice\_scheme\_plotter.Scheme method)@\spxentry{addTransition()}\spxextra{nice\_scheme\_plotter.Scheme method}}

\begin{fulllineitems}
\phantomsection\label{\detokenize{nice_scheme_plotter:nice_scheme_plotter.Scheme.addTransition}}\pysiglinewithargsret{\sphinxbfcode{\sphinxupquote{addTransition}}}{\emph{Transition\_object}}{}
Plots transition on the scheme.
\begin{quote}\begin{description}
\item[{Param}] \leavevmode
Transition\_object

\end{description}\end{quote}

\end{fulllineitems}

\index{addTransitionsPackage() (nice\_scheme\_plotter.Scheme method)@\spxentry{addTransitionsPackage()}\spxextra{nice\_scheme\_plotter.Scheme method}}

\begin{fulllineitems}
\phantomsection\label{\detokenize{nice_scheme_plotter:nice_scheme_plotter.Scheme.addTransitionsPackage}}\pysiglinewithargsret{\sphinxbfcode{\sphinxupquote{addTransitionsPackage}}}{\emph{transitionsPackage}}{}
Plots all transition from the transitions package (see more about transitionsPackage).
\begin{quote}\begin{description}
\item[{Param}] \leavevmode
transitionsPackage

\end{description}\end{quote}

\end{fulllineitems}

\index{enableLatex() (nice\_scheme\_plotter.Scheme method)@\spxentry{enableLatex()}\spxextra{nice\_scheme\_plotter.Scheme method}}

\begin{fulllineitems}
\phantomsection\label{\detokenize{nice_scheme_plotter:nice_scheme_plotter.Scheme.enableLatex}}\pysiglinewithargsret{\sphinxbfcode{\sphinxupquote{enableLatex}}}{}{}
Enables LaTeX rendering for all strings in the scheme plot.
(!) This function has to be called \sphinxstylestrong{before}
Scheme\_object.addLevel(), Scheme\_object.addLevelsPackage() methods (and analogously for add-transitions).

\end{fulllineitems}

\index{save() (nice\_scheme\_plotter.Scheme method)@\spxentry{save()}\spxextra{nice\_scheme\_plotter.Scheme method}}

\begin{fulllineitems}
\phantomsection\label{\detokenize{nice_scheme_plotter:nice_scheme_plotter.Scheme.save}}\pysiglinewithargsret{\sphinxbfcode{\sphinxupquote{save}}}{\emph{fileName=None}}{}
Saves plot to the file.
:param: fileName: filename. It is recommended to use .svg extension, for example fileName=’my\_scheme.svg’. It is
also allowed to \sphinxstylestrong{not pass} any file name (especially if there will be more than one Scheme\_object plots saved
during code operation. The Scheme class will enumerate all of it’s instances, and later save them to different
files. Example:

\fvset{hllines={, ,}}%
\begin{sphinxVerbatim}[commandchars=\\\{\}]
\PYG{g+gp}{\PYGZgt{}\PYGZgt{}\PYGZgt{} }\PYG{n}{s1} \PYG{o}{=} \PYG{n}{Scheme}\PYG{p}{(}\PYG{p}{)}
\PYG{g+gp}{\PYGZgt{}\PYGZgt{}\PYGZgt{} }\PYG{n}{s2} \PYG{o}{=} \PYG{n}{Scheme}\PYG{p}{(}\PYG{p}{)}
\PYG{g+gp}{\PYGZgt{}\PYGZgt{}\PYGZgt{} }\PYG{n}{s3} \PYG{o}{=} \PYG{n}{Scheme}\PYG{p}{(}\PYG{p}{)}
\PYG{g+gp}{\PYGZgt{}\PYGZgt{}\PYGZgt{} }\PYG{o}{.}\PYG{o}{.}\PYG{o}{.}
\PYG{g+gp}{\PYGZgt{}\PYGZgt{}\PYGZgt{} }\PYG{n}{s1}\PYG{o}{.}\PYG{n}{save}\PYG{p}{(}\PYG{p}{)}
\PYG{g+gp}{\PYGZgt{}\PYGZgt{}\PYGZgt{} }\PYG{n}{s2}\PYG{o}{.}\PYG{n}{save}\PYG{p}{(}\PYG{p}{)}
\PYG{g+gp}{\PYGZgt{}\PYGZgt{}\PYGZgt{} }\PYG{n}{s3}\PYG{o}{.}\PYG{n}{save}\PYG{p}{(}\PYG{p}{)}
\end{sphinxVerbatim}

In the result three files will be created: \sphinxcode{\sphinxupquote{scheme\_part\_1.svg}}, \sphinxcode{\sphinxupquote{scheme\_part\_2.svg}} and \sphinxcode{\sphinxupquote{scheme\_part\_3.svg}}.
It is useful when scheme splitting for many pages is needed.

\end{fulllineitems}

\index{show() (nice\_scheme\_plotter.Scheme method)@\spxentry{show()}\spxextra{nice\_scheme\_plotter.Scheme method}}

\begin{fulllineitems}
\phantomsection\label{\detokenize{nice_scheme_plotter:nice_scheme_plotter.Scheme.show}}\pysiglinewithargsret{\sphinxbfcode{\sphinxupquote{show}}}{}{}
Shows resulting scheme.

\end{fulllineitems}


\end{fulllineitems}



\chapter{Indices and tables}
\label{\detokenize{nice_scheme_plotter:indices-and-tables}}\begin{itemize}
\item {} 
\DUrole{xref,std,std-ref}{genindex}

\item {} 
\DUrole{xref,std,std-ref}{modindex}

\item {} 
\DUrole{xref,std,std-ref}{search}

\end{itemize}


\renewcommand{\indexname}{Python Module Index}
\begin{sphinxtheindex}
\let\bigletter\sphinxstyleindexlettergroup
\bigletter{d}
\item\relax\sphinxstyleindexentry{database\_reader}\sphinxstyleindexpageref{nice_scheme_plotter:\detokenize{module-database_reader}}
\indexspace
\bigletter{n}
\item\relax\sphinxstyleindexentry{nice\_scheme\_plotter}\sphinxstyleindexpageref{nice_scheme_plotter:\detokenize{module-nice_scheme_plotter}}
\end{sphinxtheindex}

\renewcommand{\indexname}{Index}
\printindex
\end{document}